\documentclass[12pt,a4paper]{article}
\usepackage[utf8x]{inputenc}
\usepackage{ucs}
\usepackage{amsmath}
\usepackage{amsfonts}
\usepackage{amssymb}
\usepackage{url}
\author{AHOUNOU Folabi Thierry}
\title{XSD : création et validation de CV}
\date{14/03/2013}
\begin{document}
	\maketitle	
	
	\newpage 	
	
	\section{Objectif}
	Le but de ce TP est de concevoir un schéma de définition d'un document XML(XML Schema Definition) permettant de décrire et de valider 		un CV. Le fichier XSD produit à l'issu du TP sera par la suite utilisé pour décrire et valider le fichier XML décrivant notre CV.
	
	\section{Analyse}
	Pour réaliser le travail demandé j'ai découpé un CV en 6 grandes parties à savoir : 
	\begin{itemize}
		\item Un titre.
		\item La civilité.
		\item La liste des formations.
		\item La liste des expériences.
		\item La liste des compétences.
		\item Et pour finir la liste des loisirs.
	\end{itemize}
	
	\section{Réalisation}
	\subsection{Le Titre du CV}
	Ici rien de particulier. Le titre du CV est basé sur une simple chaine de caractère.
	
	\subsection{La civilité}
	La civilité est composé des champs nom, prénom, âge, adresse, téléphone, email et image. Les champs noms et prénom sont des simples
	chaines de caractère. \\
	Le nom doit être en majuscule. L'expression régulière utilisée est la suivante : $ \textbf{[A-Z \-]*} $.\\
	Concernant l'âge une restriction a été également établie. En effet la valeur de l'âge doit être positive et comprise entre 16 et 75. 
	Le téléphone doit être composé de 10 chiffres séparés deux à deux par un tiret(\textbf{-});
	La restriction sur l'email a été quant à lui un plus compliqué à réaliser. J'ai finalement utilisé une expression régulière trouvée
	sur ce site : \url{http://regexlib.com}. L'expression régulière permet de matcher les chaines suivantes : foo@bar.com | 					foobar@foobar.com.au \\
	Je n'ai utilisé aucune restriction pour l'image.
	
	\subsection{La liste des formations}
	Le terme "liste des formations" résume tout ce qu'il y a à dire dans cette partie. En effet on a une formation qui est composé de
	plusieurs formations. Le format utilisé est le suivant : \\
	$
	<Formations>\\
		<Formation>...</Formation>\\
		<Formation>...</Formation>\\
	</Formations>
	$\\
	Une formation(sans s) a un attribut \textbf{id} qui est de type int, un attribut \textbf{ecole} de type string et un attribut
	\textbf{titre} pour le titre de la formation qui est de type string. La formation est aussi composée d'une date, d'une description,
	la ville où la formation a été faite et le pays.
	
	\subsection{La liste des expériences}	
	Voir partie ci dessus(La liste des formations). La réalisation est exactement la même.
	
	\subsection{La liste des compétences}
	La réalisation est semblable à la partie liste des formations et liste des expériences. La seule différence ici est qu'on peut 
	distinguer les compétences. Par exemple les compétences dans langages de programmation, les compétences dans dans les logiciels de
	bureautique. Pour faire ça j'ai juste ajouté un attribut à l'élément $ <Competences> $. L'élément $ <Competence> $ incluse dans 
	$ <Competences> $ ne possède qu'un seul attribut.
	
	\subsection{La liste des loisirs}
	La liste des loisirs est pareil comme la liste des compétences.
\end{document}